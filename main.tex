\documentclass{report}
\usepackage{subcaption}
\usepackage{tocloft}
\usepackage{graphicx}
\usepackage{float}
\usepackage{hyperref}
\usepackage{titlesec}
\usepackage{amsmath}
\usepackage{biblatex}[style=ieeetr, sorting=none, backend=biber]
\usepackage[a4paper, left=15mm, right=15mm, top=15mm, bottom=15mm]{geometry}
\addbibresource{main.bib}
\titleformat{\chapter}[display]
  {\normalfont\LARGE\bfseries}{}{0pt}{\LARGE}
\titlespacing*{\chapter}{0pt}{-50pt}{20pt}
\author{QUT Aerospace Society}
\title{Technical Report: CubeSAT}
\setcounter{tocdepth}{3}
\begin{document}    
    \maketitle
    \tableofcontents
    \chapter{CubeSAT Overview}
    % high level overview of design of the payload system
    % design functinoality of the payload system
        \section{Technical Breakdown}
            \begin{figure}[H]
                \centering
                \includegraphics[width=0.8\textwidth]{figures/CUBESAD.png}
                \caption{Block Diagram of the Payload Subsystem}
                \label{fig:payload}
            \end{figure}
            The payload subsystem of the CubeSAT project is essential for the success
            of the project and for the collection of data.
            The system is powered via a 18650 LiPo battery which is regulated to 3.3V 
            to power the barometer, IMU, Magnetometer, GPS, and microncontroller (ESP32-S3-Mini-1).
            The CubeSAT project consists of 5 PCBs with a 3D printed structure to house the components.
            \begin{itemize}
                \item Batteries
                \item Power Regulation
                \item Communication and Data Handling
                \begin{itemize}
                    \item Microcontroller PCB
                    \item Sensor PCB
                    \begin{itemize}
                        \item QMC5883L Magnetometer
                        \item BMI270 IMU
                        \item BMP388 Barometer
                        \item SAM-M10Q GPS
                        \item ESP32-S3-Mini-1
                    \end{itemize}
                \end{itemize}
                \item Communications
                \begin{itemize}
                    \item RFD900
                    \item Antennas
                \end{itemize}
                \item Communication Transmission (Antennae)
            \end{itemize}
            \subsection{Communication and Data Handling}
                \subsubsection{Microcontroller Selection}
                    The microcontroller serves as the central processing unit for the 
                    CDH system, interfacing with sensors and managing data flow.
                    The selection process shall consider the following requirements
                    \begin{itemize}
                        \item Support for I2C, and SPI communication protocols
                        \item Dual-core architecture for efficient multitasking 
                        \item UART/Serial support for programming and communication
                        \item USB bootloader availability for easy firmware updates
                        \item Integrated flash memory for code storage (optional)
                        \item Arduino IDE compatibility for ease of development
                    \end{itemize}
                    By factoring the above requirements, the following options
                    were considered
                    \begin{itemize}
                        \item ESP32-S3-Mini-1
                        \item RP2040
                        \item STM32-F103
                    \end{itemize}
                    Given the similarity in functionality, and development support between
                    all three options, the ESP32-S3-Mini-1 has been chosen for its reliability, 
                    prior experience with the team, and lack of impedance issues present in 
                    options such as the RP2040 \cite{rp2040datasheet}. Furthermore, its superior performance and efficiency
                    compared to other options make it a viable candidate for future improvements,
                    and enhancements to the project.
                    \begin{figure}[H]
                        \centering
                        \includegraphics[width=0.2\linewidth]{figures/MC_IMG.png}
                        \caption{Image of the ESP32-S3-Mini-1 Microcontroller}
                        \label{fig:mc_img}
                    \end{figure}
                \subsubsection{GPS Selection}
                    The GPS module serves as the primary source of location data for the
                    CubeSAT, providing accurate positioning information for the payload. 
                    The selection process shall consider the following requirements
                    \begin{itemize}
                        \item Accuracy
                        \begin{itemize}
                            \item Support for multi-band GNSS (L1/E1), with a < 1.5m horizontal accuracy
                            \item Position updates at a minimum rate of 10Hz for real-time tracking and navigation
                        \end{itemize}
                        \item Communication
                        \begin{itemize}
                            \item UART communication upto 115200 baud for interfacing with the microcontroller
                            \item Power efficient serial communication operation
                        \end{itemize}
                        \item Operating power
                        \begin{itemize}
                            \item 3.3 or 5V operating voltage for optimal compatibility with existing power systems
                            \item Consumption of $<$ 25mA during continuous tracking to remain within limited power budgets
                        \end{itemize}
                    \end{itemize}
                    Taking into account the above requirements, the SAM-M10Q GPS module was chosen 
                    for its high accuracy, low power consumption, and compatibility with the ESP32-S3-Mini.
                    The module provides a high update rate of 10Hz, and supports multi-band GNSS for
                    improved accuracy and reliability. Furthermore, its low power consumption of 21mA at 3.3V
                    makes it an ideal candidate for the CubeSAT project \cite{sammdatasheet}
                    \begin{figure}[H]
                        \centering
                        \includegraphics[width=0.3\linewidth, angle=290]{figures/GPS_IMG.pdf}
                        \caption{Image of the SAM-M10Q GPS Module}
                        \label{fig:gps_img}
                    \end{figure}
                \subsubsection{IMU Selection}
                    A low power, high accuracy IMU is essential for the CubeSAT project, providing
                    accurate orientation and motion data for the payload. The selection process shall
                    consider the following requirements
                    \begin{itemize}
                        \item Low cost (less than \$10 per unit)
                        \item Low power and high precision (up to $\pm$ 16g accelerometer, $\pm$ 2000 degree/s gyroscope)
                        \item Flexibility in communication (I2C, SPI)
                    \end{itemize}
                    By factoring the above requirements, the BMI270 IMU was chosen for its high accuracy,
                    low power consumption, and compatibility with the ESP32-S3-Mini. The module provides
                    a wide range of motion sensing capabilities, with a $\pm$ 16g accelerometer, and $\pm$ 2000 $\circ$/s gyroscope.
                    Furthermore, its low power consumption of 700$\mu$A during operation, i2C (up to 3.4MHz) and SPI (up to 10MHz) communication
                    interfaces, dual core architecture with a 16KB FIFO buffer for configuration and buffering,
                    Arduino libraries for easy integration \cite{bosch2023}, and low cost of \$7 to \$10 per unit make it an ideal candidate for the CubeSAT project \cite{bosch2023}
                    % \begin{figure}[H]
                    %     \centering
                    %     \includegraphics[width=0.2\linewidth]{figures/IMU_IMG.png}
                    %     \caption{Image of the BMI270 IMU Module}
                    %     \label{fig:imu_img}
                    % \end{figure}
                \subsubsection{Magnetometer Selection}
                    A space, cost, and energy efficient magnetometer is essential for the CubeSAT project,
                    providing accurate magnetic field data for the payload. The selection process shall consider
                    the following requirements
                    \begin{itemize}
                        \item Low cost (less than \$5 per unit)
                        \item Low power and high precision (up to $\pm$ 8 gauss)
                        \item Flexibility in communication (I2C, SPI)
                        \item Small form factor for easy integration
                        \item Wide operating temperature range (-40 to 85 $^\circ$C)
                        \item Low power consumption (less than 100$\mu$A)
                    \end{itemize}
                    By factoring the above requirements, the QMC5883L magnetometer was chosen for its high accuracy,
                    low power consumption, and compatibility with the ESP32-S3-Mini. The module provides a wide range of magnetic field sensing capabilities,
                    with a $\pm$ 8 gauss range, and 0.2 gauss resolution. Furthermore, its low power consumption of 100$\mu$A during operation, I2C (up to 400kHz) communication interface,
                    while it lacks SPI support, its small form factor, wide operating temperature range \cite{qmc5883}, and low cost of \$2 to \$5 per unit make it an ideal candidate for the CubeSAT project \cite{qmc5883}
                    \begin{figure}[H]
                        \centering
                        \includegraphics[width=0.2\linewidth]{figures/MAG_IMG.png}
                        \caption{Image of the QMC5883L Magnetometer Module}
                        \label{fig:mag_img}
                    \end{figure}
                \subsubsection{Barometer Selection}
                    A space, cost, and energy efficient barometer is essential for the CubeSAT project,
                    providing accurate pressure data for the payload. The selection process shall consider
                    the following requirements
                    \begin{itemize}
                        \item Low cost (less than \$5 per unit)
                        \item Wide operating temperature range (-40 to 85 $^\circ$C)
                        \item Wide operating pressure range (30 to 125 kPa)
                        \item High absolute accuracy ($\pm$ 50 Pa)
                    \end{itemize}
                    Three barometers were considered for the CubeSAT project, the ICP10101, S-THP-01A, and
                    the BMP388. However, the BMP388's superior accuracy (50Pa vs 100Pa for the ICP10101),
                    support for a wider range of communication protocols (i2C and SPI vs i2C only for the ICP10101, and 
                    RS485 for the S-THP-01A), and lower cost (\$1.65 for the BMP388 vs \$6.08 for the ICP10101, and \$55.09 for the S-THP-01A)
                    make it an ideal candidate for the CubeSAT project \cite{bosch2023}
                    \begin{figure}[H]
                        \centering
                        \includegraphics[width=0.2\linewidth, angle=260]{figures/BARO_IMG.pdf}
                        \caption{Image of the BMP388 Barometer Module}
                        \label{fig:baro_img}
                    \end{figure}
            % \subsection{Power}
            %     Power is supplied to the payload system via two 18650 LiPo batteries in parallel.
            %     The batteries are regulated to 3.3V via a pair of TPS63060DSCR buck-boost converters, regulating
            %     voltage to 5V and 3.3V for the payload system, with an STC3100IQT battery monitor to monitor battery
            %     voltage and current.
            %     \subsubsection{Battery}
            %         The batteries are connected in parallel to provide redundancy and increase capacity.
            %         \begin{figure}[H]
            %             \centering
            %             \includegraphics[width=0.2\linewidth]{figures/BATTERY.png}
            %         \end{figure}
            %     \subsubsection{Buck-Boost Converter}
            %     \subsubsection{Battery Monitor}
            \subsection{Radio Modem}
                The RFD900ux STM radio modem is used for communication between the CubeSAT and the ground station.
                The modem operates on the 915MHz frequency band, with a maximum output power of 1W, and a range of up to 40km.
                The modem supports serial communication at up to 115200 baud, with a power consumption of 1.5W during transmission,
                and 0.5W during reception.
                Simple integration using the MAVLink Protocol, and 
                support for AES-128 encryption allows for easy integration and secure communication between the CubeSAT and the ground station.
            \subsection{Antenna}
                The antenna system consists of a pair of dipole antennas, one for the CubeSAT, and one for the ground station.
                The antennas are tuned to the 915MHz frequency band, with a gain of 2.15dB, and a VSWR of 1.5:1.
                The antennas are connected to the RFD900 radio modem via SMA connectors, providing a reliable and efficient
                communication link between the CubeSAT and the ground station.
    \chapter{Design Implementation}
        \section{Power PCB Design}
            The power PCB is designed to regulate power from the batteries to the system of PCBs in the CubeSAT.
            The PCB consists of two TPS63060DSCR buck-boost converters, an STC3100IQT battery monitor, and a pair of 18650 LiPo batteries.
            The TPS63060DSCR buck-boost converters regulate voltage to 5V and 3.3V for the payload system, with the STC3100IQT battery monitor
            monitoring battery voltage and current. The buck-boost converter design follows the recommended configuration in the datasheets
            for a 3.3V and 5V output, where the voltage divider resistors were selected as follows
            \begin{figure}[H]
                \begin{flalign*}
                    R_1 &= R_2 \left(\frac{V_{\text{OUT}}}{V_{\text{FB}}} - 1 \right) \tag{\text{From datasheet}}\\
                \end{flalign*}
                \caption{Voltage Divider Calculation}
                \label{equ:volt_div}
            \end{figure}
            Letting $R_2 = 200\text{k}\Omega$, the voltage divider resistors were calculated as follows\\
            \begin{minipage}{0.45\linewidth}
                \begin{flalign*}
                    \intertext{For 3.3V output}
                    R_1 &= 200\text{k} \Omega \left(\frac{3.3V}{0.5V} - 1 \right)\\
                    R_1 &= 200\text{k} \Omega \left(6.6 - 1 \right)\\
                    \phantom{R_1} &= 1120\text{k}\Omega \approx 1.1\text{M} \Omega
                \end{flalign*}
            \end{minipage}
            \begin{minipage}{0.45\linewidth}
                \begin{flalign*}
                    \intertext{For 5V output}
                    R_1 &= 200\text{k} \Omega \left(\frac{5V}{0.5V} - 1 \right)\\
                    R_1 &= 200\text{k} \Omega \left(10 - 1 \right)\\
                    \phantom{R_1}&= 1800\text{k} \Omega = 1.8\text{M} \Omega
                \end{flalign*}
            \end{minipage}\\
            The batteries are connected in parallel to provide redundancy and increase capacity. The PCBs were designed using Altium Designer,
            and each component was integrated using the recommended configuration in the datasheets. 
            The resultant schematic and PCB layout is provided below.
            \begin{figure}[H]
                \centering
                \includegraphics[width=0.6\linewidth]{figures/PWR_SCHM.png}
                \caption{Power Regulation Schematic}
                \label{fig:power}
            \end{figure}
            The battery monitor provides a voltage and current reading for the batteries, allowing for
            real-time monitoring of battery health and performance. The configuration follows the recommended
            configuration in the datasheet \cite{STC3100} (Figure \ref{fig:battery_monitor}).
            \begin{figure}[H]
                \centering
                \includegraphics[width=0.6\linewidth]{figures/PWR_BM.png}
                \caption{Battery Monitor Configuration}
                \label{fig:battery_monitor}
            \end{figure}
        \section{Communications PCB Design}
            The communications PCB is designed to interface with the RFD900ux STM radio modem to provide communication
            between the CubeSAT and the ground station. The PCB consists of an RFD900ux STM radio modem, and
            an RFDesign Flex1-UX antenna. The combination of the RFD900ux STM radio modem and the RFDesign Flex1-UX antenna
            provides a reliable and efficient communication link between the CubeSAT and the ground station while 
            avoiding bottlenecks in data transmission due to their similar frequency bands and the low return loss of the antenna.
        \section{Command and Data Handling Design}
            \subsection{PCB Design}
                % The communication PCB is designed to interface with the RFD900 radio modem, and the ESP32-S3-Mini microcontroller.
                \begin{figure}[H]
                    \centering
                    \includegraphics[width=0.4\linewidth]{figures/MC_PINS.png}
                    \caption{ESP32S3 Microcontroller Pin Configuration}
                    \label{fig:mc_pins}
                \end{figure}
                Using the sample configuration specified in the datasheet \ref{fig:mc_example} \cite{esp32s3mini}, 
                the ESP32-S3-Mini-1 was integrated into the CDH PCB in order to interact with 
                the sensors and radio through the I2C and SPI communication protocols.
                It must be noted that the ESP32-S3-MINI-1 was utilized for cost efficiency and had the keepout zone cut off to accomodate the 
                specification. However, redesigns were not necessary as the pin configuration is identical to the ESP32-S3-MINI-1U.
                \begin{figure}[H]
                    \centering
                    \includegraphics[width=0.4\linewidth]{figures/ESP32S3_EXAMPLE.png}
                    \caption{ESP32S3 Microcontroller Example Configuration}
                    \label{fig:mc_example}
                \end{figure}
                \begin{figure}[H]
                    \centering
                    \includegraphics[width=0.4\linewidth, angle=270]{figures/BARO_DS.pdf}
                    \caption{Barometer Datasheet Configuration}
                    \label{fig:baro_ds}
                \end{figure}

                words
                \begin{figure}[H]
                    \centering
                    \includegraphics[width=0.4\linewidth]{figures/GPS_DS.pdf}
                    \caption{GPS Datasheet Configuration}
                    \label{fig:gps_ds}
                \end{figure}

                words
                \begin{figure}[H]
                    \centering
                    \includegraphics[width=0.5\linewidth, angle=270]{figures/IMU_DS.pdf}
                    \caption{IMU Datasheet Configuration}
                    \label{fig:imu_ds}
                \end{figure}
                words
                \begin{figure}[H]
                    \centering
                    \includegraphics[width=0.8\linewidth]{figures/CDH_SCHM.png}
                    \caption{Final CDH Schematic}
                    \label{fig:cdh_schm}
                \end{figure}
            \subsection{Data Handling}
                \begin{figure}[H]
                    \centering
                    \includegraphics[width=0.2\linewidth]{figures/CDH_CODE.png}
                    \caption{Data Handling Program Flowchart}
                    \label{fig:cdh}
                \end{figure}
        
    \chapter{Design Verification and Validation}
        \section{Payload Subsystem}
            \subsection{Microcontroller}
                The data handling code was first tested for functionality
                on a development board before being ported to the 
                ESP32-S3-Mini-1. The code was tested for functionality
                and performance, with the microcontroller interfacing
                with the sensors.
        \chapter{System Requirements Compliance}
    \printbibliography
\end{document}
